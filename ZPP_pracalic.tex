\documentclass[licencjacka]{pracamgr}
\usepackage{polski}
\usepackage[utf8]{inputenc}
\usepackage[table]{xcolor}
\usepackage{array}
\usepackage{amssymb}
\usepackage{amsmath}
\usepackage{amsthm}
\usepackage[pdftex]{graphicx}
\usepackage{underscore}
\usepackage{hyperref}

\author{autor1
autor2}

\nralbumu{123456
234567}

\title{tytuł}

\tytulang{English title}

\kierunek{Informatyka}

\opiekun{dra Roberta Dąbrowskiego\\
Pion Zastępcy Kanclerza ds. Informatycznych}

\date{??? 2015}

\dziedzina{ 
11.0 Matematyka, Informatyka:\\
11.1 Informatyka\\ 
}

\klasyfikacja{klasyfikacja - nie wiem, czy tu w ogóle występuje}
  
\keywords{słowa kulczowe - nie wiem, czy tu w ogóle występują}

%\newtheorem{defi}{Definicja}[section]


\begin{document}
\maketitle

\begin{abstract}
Tu będzie abstract (skrót)
\end{abstract}


\tableofcontents

\chapter*{Wprowadzenie}
\addcontentsline{toc}{chapter}{Wprowadzenie}

\begin{thebibliography}{99}
\addcontentsline{toc}{chapter}{Bibliografia}

\bibitem[abc]{ABC} Książka 1

\bibitem[def]{DEF} Link 2

\end{thebibliography}




\end{document}
